%%%%%%%%%%%%%%%%%%%%%%%%%%%%%%%%%%%%%%%%%%%%%%%%%%%%%%%%%%%%%%%%%%%%%%%%%%%%%%
\documentclass[12pt,hidelinks]{article}

% 1. Load LaTeX packages
\usepackage{fontspec}
\usepackage{geometry}
\usepackage{lastpage}
\usepackage{fancyhdr}
\usepackage{hyperref}
\usepackage{amsmath}
\usepackage{amsthm}
\usepackage{xunicode}
\usepackage{listings}
\usepackage{color}
\usepackage{amssymb}

% 2. Define page dimensions and spacing
\geometry{top=1in, bottom=1in, left=1in, right=2in, marginparsep=4pt,
          marginparwidth=1in}
\setlength{\parindent}{0pt}
\setlength{\parskip}{12pt}

% 3. Set header, footer, and bibliography
\renewcommand{\headrulewidth}{0pt}
\pagestyle{fancyplain}
\fancyhf{}
\lfoot{}
\rfoot{page \thepage\ of \pageref{LastPage}}
\bibliographystyle{acm}

% 4. Set fonts for the document
\defaultfontfeatures{Mapping=tex-text}
\setromanfont{YaleNew}

% 5. Define custom code for book environments and commands
\DeclareMathOperator*{\argmin}{arg\,min}
\DeclareMathOperator*{\argmax}{arg\,max}
\newcommand{\code}[1]{\texttt{#1}}
\newcommand{\pkg}[1]{\textbf{#1}}

% 6. Define custom code for book environments and commands
\definecolor{verbgray}{gray}{0.9}
\definecolor{verbgray2}{gray}{0.975}

\lstnewenvironment{rcode}{%
  \lstset{backgroundcolor=\color{verbgray},
  frame=single,
  framerule=0pt,
  basicstyle=\ttfamily,
  keepspaces=true,
  columns=fullflexible}}{}

\lstnewenvironment{rres}{%
  \lstset{backgroundcolor=\color{verbgray2},
  frame=single,
  framerule=0pt,
  basicstyle=\ttfamily,
  keepspaces=true,
  columns=fullflexible}}{}

% 7. Define numbering scheme for equations (only needed for handout).
\numberwithin{equation}{section}
\setcounter{section}{2}

%%%%%%%%%%%%%%%%%%%%%%%%%%%%%%%%%%%%%%%%%%%%%%%%%%%%%%%%%%%%%%%%%%%%%%%%%%%%%%
\begin{document}

{\LARGE Handout 02: Simple Linear Regression}

\vspace*{18pt}

Last time, I said that I would be using note from my textbook this semester
to teach the course. Today is one exception because I think it is useful to
start with a bit of review that is not included in the text. For this, I think
it is best to just jump right into the notes rather than explaining this to
you in detail. Hopefully the material today will come quickly. If this seems
overwhelming, today would be a good time to talk with me about your background
and the course going forward.

% \renewcommand{\section}[2]{}%
% \vspace{12pt}
% \textbf{References}
% \bibliography{bibfile}

%%%%%%%%%%%%%%%%%%%%%%%%%%%%%%%%%%%%%%%%%%%%%%%%%%%%%%%%%%%%%%%%%%%%%%%%%%%%%%
\newpage

\textbf{LAB QUESTIONS}

\vspace*{0pt}

\begin{enumerate}
\item To start, download and open the \verb|class02.Rmd| file in RStudio.
Follow the script until you get to the section that asks you to return to
these notes.
\item Last time we started with the basic idea of statistical learning. We
observe pairs $(x_i, y_i)$ and want to construct a function $f(x)$ from this
training data that does a good job of predicting future values of $y_i$ given
new values of $x_i$. One of the simplest such models for predicting a
continuous response $y$ is simple linear regression. Visually this corresponds
to fitting a linear function $f$ to the data such that:
\begin{align}
f(x_i) &= a + b \cdot x_i.
\end{align}
Where the parameters $a$ (the intercept) and $b$ (the intercept) are
\textit{learned} from the data. Write down, symbolically, what the mean
squared loss function is of using the above $f$ to predict the values $y_i$.
\item We are going to simplify things further by removing the intercept term
$a$ from the model and assuming that we have only:
\begin{align}
f(x_i) &= b \cdot x_i.
\end{align}
Taking the equation you had from the previous question, write down the loss
function for the new value of $f$. Take the derivative with respect to $b$
and set it equal to zero. Can you find a formula for $b$ that minimizes the
loss function?
\item Taking the second derivative of the loss function, prove that you found
a global minimizer in the previous question rather than a saddle point or
maximum.
\item We typically write the learned parameters in a model with a `hat'. So
the slope you computed above becomes $\widehat{b}$. Can you re-write $\widehat{b}$
such that the estimator is written a weighted sum of the values $y_i$?
\item So far, we have made no assumptions about the `true' nature of the
relationship between $x$ and $y$. Assume that we can write:
\begin{align}
y_i &= b \cdot x_i + \epsilon_i
\end{align}
For some term $\epsilon_i$ known as the \textit{error term}. Plugging this into
your equation for $\widehat{b}$, can you argue that $\widehat{b}$ will be close
to b if the error terms are small?
\item Return to the R code to complete today's lab.
\end{enumerate}

\end{document}

