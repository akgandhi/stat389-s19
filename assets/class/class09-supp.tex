%%%%%%%%%%%%%%%%%%%%%%%%%%%%%%%%%%%%%%%%%%%%%%%%%%%%%%%%%%%%%%%%%%%%%%%%%%%%%%
\documentclass[12pt,hidelinks]{article}

% 1. Load LaTeX packages
\usepackage{fontspec}
\usepackage{geometry}
\usepackage{lastpage}
\usepackage{fancyhdr}
\usepackage{hyperref}
\usepackage{amsmath}
\usepackage{amsthm}
\usepackage{xunicode}
\usepackage{listings}
\usepackage{color}
\usepackage{amssymb}

% 2. Define page dimensions and spacing
\geometry{top=1in, bottom=1in, left=1in, right=2in, marginparsep=4pt,
          marginparwidth=1in}
\setlength{\parindent}{0pt}
\setlength{\parskip}{12pt}

% 3. Set header, footer, and bibliography
\renewcommand{\headrulewidth}{0pt}
\pagestyle{fancyplain}
\fancyhf{}
\lfoot{}
\rfoot{page \thepage\ of \pageref{LastPage}}
\bibliographystyle{acm}

% 4. Set fonts for the document
\defaultfontfeatures{Mapping=tex-text}
\setromanfont{YaleNew}

% 5. Define custom code for book environments and commands
\DeclareMathOperator*{\argmin}{arg\,min}
\DeclareMathOperator*{\argmax}{arg\,max}
\newcommand{\code}[1]{\texttt{#1}}
\newcommand{\pkg}[1]{\textbf{#1}}

% 6. Define custom code for book environments and commands
\definecolor{verbgray}{gray}{0.9}
\definecolor{verbgray2}{gray}{0.975}

\lstnewenvironment{rcode}{%
  \lstset{backgroundcolor=\color{verbgray},
  frame=single,
  framerule=0pt,
  basicstyle=\ttfamily,
  keepspaces=true,
  columns=fullflexible}}{}

\lstnewenvironment{rres}{%
  \lstset{backgroundcolor=\color{verbgray2},
  frame=single,
  framerule=0pt,
  basicstyle=\ttfamily,
  keepspaces=true,
  columns=fullflexible}}{}

% 7. Define numbering scheme for equations (only needed for handout).
\numberwithin{equation}{section}
\setcounter{section}{9}

%%%%%%%%%%%%%%%%%%%%%%%%%%%%%%%%%%%%%%%%%%%%%%%%%%%%%%%%%%%%%%%%%%%%%%%%%%%%%%
\begin{document}

{\LARGE Handout 09: Logistic Regression (Supplement)}

\vspace*{18pt}

On the original handout I gave you the following two formulas:
\begin{align*}
p_i &= \frac{e^{x_i^t b}}{e^{x_i^t b} + 1} \\
1 - p_i &= \frac{1}{1 + e^{x_i^t b}}
\end{align*}
I should have noted that if you multiple both of these by:
\begin{align*}
1 &= \frac{e^{-x_i^t b}}{e^{-x_i^t b}}
\end{align*}
You will get the alternative forms:
\begin{align*}
p_i &= \frac{1}{1 + e^{-x_i^t b}} \\
1 - p_i &= \frac{e^{-x_i^t b}}{e^{-x_i^t b} + 1}
\end{align*}
So if you multiply $x_i^t b$ by negative one, you flip the equations for $p_i$
and $(1 - p_i)$. This is what I meant by logistic regression being \textit{symmetric}
between $0$ and $1$. If you flip the signs of the regression vector $\beta$ you will
have the logistic model with the two classes flipped.

Beyond the demonstration that show this kind of symmetry, the alternative forms
are also useful for deriving the Hessian matrix on the problems from the handout.

\end{document}

