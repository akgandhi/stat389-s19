%%%%%%%%%%%%%%%%%%%%%%%%%%%%%%%%%%%%%%%%%%%%%%%%%%%%%%%%%%%%%%%%%%%%%%%%%%%%%%
\documentclass[12pt,hidelinks]{article}

% 1. Load LaTeX packages
\usepackage{fontspec}
\usepackage{geometry}
\usepackage{lastpage}
\usepackage{fancyhdr}
\usepackage{hyperref}
\usepackage{amsmath}
\usepackage{amsthm}
\usepackage{xunicode}
\usepackage{listings}
\usepackage{color}
\usepackage{amssymb}

% 2. Define page dimensions and spacing
\geometry{top=1in, bottom=1in, left=1in, right=2in, marginparsep=4pt,
          marginparwidth=1in}
\setlength{\parindent}{0pt}
\setlength{\parskip}{12pt}

% 3. Set header, footer, and bibliography
\renewcommand{\headrulewidth}{0pt}
\pagestyle{fancyplain}
\fancyhf{}
\lfoot{}
\rfoot{page \thepage\ of \pageref{LastPage}}
\bibliographystyle{acm}

% 4. Set fonts for the document
\defaultfontfeatures{Mapping=tex-text}
\setromanfont{YaleNew}

% 5. Define custom code for book environments and commands
\DeclareMathOperator*{\argmin}{arg\,min}
\DeclareMathOperator*{\argmax}{arg\,max}
\newcommand{\code}[1]{\texttt{#1}}
\newcommand{\pkg}[1]{\textbf{#1}}

% 6. Define custom code for book environments and commands
\definecolor{verbgray}{gray}{0.9}
\definecolor{verbgray2}{gray}{0.975}

\lstnewenvironment{rcode}{%
  \lstset{backgroundcolor=\color{verbgray},
  frame=single,
  framerule=0pt,
  basicstyle=\ttfamily,
  keepspaces=true,
  columns=fullflexible}}{}

\lstnewenvironment{rres}{%
  \lstset{backgroundcolor=\color{verbgray2},
  frame=single,
  framerule=0pt,
  basicstyle=\ttfamily,
  keepspaces=true,
  columns=fullflexible}}{}

% 7. Define numbering scheme for equations (only needed for handout).
\numberwithin{equation}{section}
\setcounter{section}{11}

%%%%%%%%%%%%%%%%%%%%%%%%%%%%%%%%%%%%%%%%%%%%%%%%%%%%%%%%%%%%%%%%%%%%%%%%%%%%%%
\begin{document}

{\LARGE Handout 11: Regression Splines}

\vspace*{18pt}

\textbf{Non-linearity}

Linear regression has excellent theoretical properties and,
as we have seen, can be readily computed from observed data.
Using ridge regression and principal component analysis we
can tune these models to optimize for predictive error loss.
Indeed, linear models are used throughout numerous fields for
predictive and inferential models. One situation in which
linear models begin to perform non-optimally is when the
relationship between the response $y$ and the data is
not linear nor can it be approximated closely by a linear
relationship.

As an example of a non-linear model consider observing a
variable $y_i$ governed by
\begin{align}
y_i &= \cos(\beta_1 \cdot x_i) + e^{-x_i \cdot \beta_2} + \epsilon_i \label{nonlinear_example}
\end{align}
for some scalar value $x_i$, unknown constants $\beta_1$
and $\beta_2$, and the random noise variable $\epsilon_i$.
A common approach for estimating the unknown parameters given
a set of observations is to again minimize the sum of squared
residuals. This sum is a well-defined function over the set of
allowed $\beta_j$'s and often, as in this case, twice differentiable.
While there is no analogous closed-form solution to the linear
case, the minimizing estimate values can usually be found using
a general purpose first, or second-order optimization technique.
This approach is known as \textit{non-linear least squares} and
has significant theoretical guarantees over a wide class of
problem formulations. \index{non-linear least squares}

\index{non-parametric regression}
What happens when we do not know a specific formula for $y_i$
that can be written down in terms of a small set of unknown
constants $\beta_j$? Models of the form seen in Equation~\ref{nonlinear_example}
often arise in engineering and science applications where the
specific causal mechanism for the response $y_i$ is well
understood. In statistical learning this is rarely the case.
More often we just know that the expected value is equal to some function
\begin{align}
\mathbb{E} y_i &= f(x_i) \label{nonparam_def}
\end{align}
holds for some unknown function $f$. We may suspect that
$f$ has some general properties; depending on the application it may
be reasonable to assume that $f$ is continuous, has a bounded derivative,
or is monotonically increasing in $x_i$. As we do not know a specific
formula for $f$ in terms of parameters $\beta_j$, the model given in
Equation~\ref{nonlinear_example} is known as non-parametric regression.
Common estimators for estimating the non-parametric regression function
$f$ are the topic of this chapter.


In parametric regression it is clear that a point estimator will yield
a single prediction $\widehat{\beta}$ inside of $\mathbb{R}^p$ for the
unknown regression vector. For non-parametric regression it is not even
clear what an estimator $\widehat{f}$ would look like. Two methods we
evaluate will produce an explicit formula in terms of $x_i$ as a prediction
of $y_i$. The other two instead provide an algorithm for computing
$\widehat{f}(x)$ for any input value of $x$. While it is
possible to do this for a large set of $x$'s, these techniques
will not yield an estimated
parametric model such as that given in Equation~\ref{nonlinear_example}.
The added computational time for predictions, which often require
non-trivial computations for every value of $x$ should be taken into
account when considering the value added to the predictive performance
by a non-parametric model. We will discuss techniques for minimizing
the burden of the prediction time.

As with most predictive models, non-parametric regression techniques
have tuning parameters to control the trade-off between variance and
bias. Often this can be visualized in terms of the smoothness of the
function $\widehat{f}$. If the regression function is penalized from
changing too fast, this reduces the variance in the estimates but
introduces additional bias for values at $x_i$ where it \textit{should}
vary more. Likewise, if the function is allowed to change rapidly this
will generally give nearly unbiased estimates at the price of a high
variance in the estimated values. Much of our discussion about non-parametric
regression will focus on methods for setting tuning parameters and the
performance of these methods on various types of data.

\textbf{Basis expansion}

Recall that linear regression requires
only that the relationship with respect to the parameter $\beta_j$ be linear.
The terms multiplied by each parameter may be any known quantity
derivable from the data matrix $X$. For example, with a scalar value
of $x_i$ both
\begin{align}
y_i &= \sum_{k = 0}^K x_i^k \cdot \beta_j + \epsilon
\end{align}
and
\begin{align}
y_i &= \sum_{k = 1}^K sin(x_i / (2\pi K)) \cdot \beta_j + \epsilon
\end{align}
are valid linear regression models. The first corresponds to the first
$K$ terms of the polynomial basis and the second to the first $K$ odd
terms of the Fourier basis. If we construct a new matrix $Z$ consisting
of columns that are copies of $x$ taken to various powers or to varying
applications of the sine function, an estimate of the relationship
between $y$ and $x$ can be determined using the standard techniques
for calculating a linear regression model. In general we model the
relationship
\begin{align}
y_i &= \sum_{k = 1}^K B_{k, K}(x_i) \cdot \beta_j + \epsilon
\end{align}
for some basis function $B_{k,K}$. This method is known as a
\textit{basis expansion}.

\textbf{Regression splines}

Here we discuss an important set of basis functions, known
as splines, that are particularly well-suited to this task. While technically
just a specific application of basis expansion, the derivation of the
splines is subtle enough and their application sufficiently important to
warrant a separate treatment of their form and usage.

\index{Fourier basis}
When using a polynomial or Fourier basis to represent a non-linear function,
small changes in a coefficient will lead to changes in the predicted values
at every point of the unknown regression function $f$. The global nature
of the estimation problem in these cases leads to poor local performance
in the presence of high noise variance or with regression functions $f$
that have many critical points. Local regression, as we have seen, offers
a solution to this problem by using a relatively small basis expansion
locally at each point. A downside of this approach, however, is that it
requires fitting a regression model for every desired point where a
prediction is needed. The computational load of this calculation can
become considerable and in the case of larger datasets, the requirement
that all training data must be accessible to run this regression may also
limit its application. Fortunately there is a way of doing basis
expansion that mimics the primary benefits of local regression.

We start by picking a point $k$ within the range of the data points $x_i$.
A natural choice would be the median or mean of the data. In lieu of
a higher-order polynomial fit, imagine fitting two linear polynomials
to the data: one for points less than $k$ and another for points greater
than $k$. Using indicator functions, we can describe this approach
with a specific basis expansion, namely
\begin{align}
B_0(x) &= I(x \leq k) \\
B_1(x) &= x \cdot I(x \leq k) \\
B_2(x) &= I(x > k) \\
B_3(x) &= x \cdot I(x > k).
\end{align}
It will be useful going forward to re-parameterize this in terms of a
baseline intercept and slope for $x \leq k$ and changes in these values
for points $x > k$
\begin{align}
B_0(x) &= 1 \label{tp_basis_orig_start} \\
B_1(x) &= x \\
B_2(x) &= I(x > k) \\
B_3(x) &= (x - k) \cdot I(x > k). \label{tp_basis_orig_end}
\end{align}
It is left as an exercise to show that these are equivalent bases.

A shortcoming of the space spanned by these splines is that at the
point $k$, known as a \textit{knot}, the predicted values will
generally not be continuous. It is possible to modify our
original basis to force continuity at the knot $k$ by removing
the secondary intercept described by $B_2(x)$ in
Equations~\ref{tp_basis_orig_start}--\ref{tp_basis_orig_end}.
The basis now becomes
\begin{align}
B_0(x) &= 1 \\
B_1(x) &= x \\
B_2(x) &= (x - k) \cdot I(x > k).
\end{align}
Notice that forcing one constraint, continuity at $k$, has reduced
the degrees of freedom by one, from $4$ down to $3$. How might we
generalize this to fitting separate quadratic term on the two halves
of the data? One approach would be to use the basis functions
\begin{align}
B_0(x) &= 1 \label{tp_basis_quad_start} \\
B_1(x) &= x \\
B_2(x) &= x^2 \\
B_3(x) &= (x - k) \cdot I(x > k) \\
B_4(x) &= (x - k)^2 \cdot I(x > k). \label{tp_basis_quad_end}
\end{align}
The number of parameters here works out correctly; we have two
quadratic polynomials ($2 \times 3$) minus one constraint, for a
total of $6-1=5$ degrees of freedom. What will a function look like
at the knot $k$ using the basis from
Equations~\ref{tp_basis_quad_start}--\ref{tp_basis_quad_end}?
It will be continuous at the knot but is not constrained to have
a continuous derivative at the point. This is easy to accomplish,
however, by removing the $B_3(x)$ basis.
Notice that once again the inclusion of an additional constraint,
a continuous first derivative, reduces the degrees of freedom by
one.

Defining the positive part function $(\cdot)_{+}$ as
\begin{align}
(x)_{+} &= \begin{cases} x, & x \geq 0 \\ 0, & \text{otherwise} \end{cases}
\end{align}
we may generalize to an arbitrarily large polynomial of order $M$ by using
the basis
\begin{align}
B_0(x) &= 1 \\
B_j(x) &= x^j, \quad j = 1, \ldots, M \\
B_{M + 1}(x) &= (x - k)_{+}^M
\end{align}
This basis results in a function with continuous derivatives of orders
$0$ through $M-1$. We can further generalize this by considering a set
of $P$ knots $\{ k_p \}_{p = 1}^P$, given by
\begin{align}
B_0(x) &= 1 \label{tp_basis_start} \\
B_j(x) &= x^j, \quad j = 1, \ldots, M \\
B_{M + p}(x) &= (x - k_p)_{+}^M, \quad p = 1, \ldots, P \label{tp_basis_end}
\end{align}
Equations~\ref{tp_basis_start}--\ref{tp_basis_end} defines the
\textit{truncated power basis} of order $M$. \index{truncated power basis}
It yields piecewise $M$th
order polynomials with continuous derivatives of order $0$ through $M-1$.
Note that once again the degrees of freedom math works out as expected.
There are $P+1$ polynomials of order $M$ and $P$ sets of $M$ constraints;
the truncated power basis has $(P+1)(M+1) - PM$, or $1+M+P$, free parameters.

Now that we have defined these basis functions, we can fit a
regression model to learn the representation of the unknown function
$f(x)$ by minimizing the sum of squared residuals over all functions
spanned by this basis. This is equivalent to the basis expansion
we used at the start of these notes. When used over the spline basis,
the resulting estimator is known as a \textit{regression spline}.
\index{regression spline}
As with any basis expansion, we can compute the solution by explicitly
constructing a design matrix $G$ as
\begin{align}
G_{i,j} &= B_{j-1}(x_i), \quad i = 1, \ldots n, \, j = 1, \ldots, 1 + M + P. \label{trun_basis_g}
\end{align}
Then, to calculate $\widehat{f}(x_{0})$, we simply compute the
basis expansion at $x_{new}$
\begin{align}
g_i &= B_{j-1}(x_0), \quad i = 1, \ldots n, \, j = 1, \ldots, 1 + M + P.
\end{align}
The regression spline can be written in this basis using a
vector $\beta \in \mathbb{R}^{1 + M + P}$
\begin{align}
\widehat{f}(x) &= \sum_j^{1 + M + P} \widehat{\beta_j} B_{j - 1}(x)
\end{align}
where $\widehat{\beta}$ is given by
\begin{align}
\widehat{\beta} &= (G^t G)^{-1} G^t y.
\end{align}
We can then compute $\widehat{f}$ this for any new point $x_{0}$, thus
providing an estimate of the entire function $f$.

By far the most commonly used truncated power basis functions are those with
$M$ equal to three. These are justified by the empirical evidence that higher
order rarely offer performance gains and that human observers are unable
to detect changes in the third derivative of a function (the idea being
that you will not be able to point out the knots in a cubic spline).


\end{document}

