\documentclass[12pt]{article}

\usepackage{fontspec}
\usepackage{geometry}
\usepackage{lastpage}
\usepackage{fancyhdr}
\usepackage{hyperref}

\geometry{top=1in, bottom=1in, left=1in, right=1in, marginparsep=4pt, marginparwidth=1in}

\renewcommand{\headrulewidth}{0pt}
\pagestyle{fancyplain}
\fancyhf{}
\lfoot{As of 2019-01-07}
\rfoot{page \thepage\ of \pageref{LastPage}}

\setlength{\parindent}{0pt}
\setlength{\parskip}{0pt}

% \setromanfont [Ligatures={Common}, Numbers={OldStyle}, Variant=01,
%  BoldFont={LinLibertine_RB.otf},
%  ItalicFont={LinLibertine_RI.otf},
%  BoldItalicFont={LinLibertine_RBI.otf}
%  ]{LinLibertine_R.otf}

\usepackage{tikz}
\def\checkmark{\tikz\fill[scale=0.4](0,.35) -- (.25,0) -- (1,.7) -- (.25,.15) -- cycle;}

\usepackage{xunicode}
\defaultfontfeatures{Mapping=tex-text}

\setromanfont{YaleNew}

\begin{document}

\begin{center}
{\bf MATH/STAT 389: Statistical Learning, Spring 2019} \\
Tuesday, Thursday 09:00-10:15 \quad JPSN G24
\end{center}

\bigskip

\noindent
\begin{tabular}{ l l }
{\bf Instructor:} &  {\bf Taylor Arnold} \\
E-mail: & \href{mailto:tarnold2@richmond.edu}{tarnold2@richmond.edu} \\
Office: & Jepson Hall, Rm 218 \\
Office hours: & By appointment (see course website for \texttt{calend.ly} link)
\end{tabular}

\vspace{0.5cm}

\textbf{Description:} \vspace{6pt}

Computational statistics and statistical algorithms for building predictive
models from large data sets. Topics include model complexity, hyper-parameter
tuning, over- and under-fitting, and the evaluation of predictive performance.
Models covered include linear regression, penalized regression, additive
models, gradient-boosted trees, and neural networks. Applications are drawn
from many areas, with a particular focus on processing unstructured text and
image corpora.

\bigskip

\textbf{Computing:} \vspace{6pt}

To facilitate your ability to actually \textit{do} statistics, most class
meetings will involve some form of computing. No prior programming experience
is assumed or required.

\medskip

We will use the \textbf{R} programming environment throughout the
semester. It is freely available for all major operating systems and
is pre-installed on many campus computers. You can download it and
all supporting files for your own machine via these links:
\begin{center}
\url{https://cran.r-project.org/} \\
\url{https://www.rstudio.com/}
\end{center}
You are required to bring a laptop to each class meeting with R installed
and running. This requires that you have a computer with an up-to-date
version of macOS, Windows, or Linux (iPads and Chromebooks will not suffice).
If this is not possible, or becomes a problem during the semester, it is your
responsibility to inform me as soon as possible so that we can find an
alternative solution.

\bigskip

\textbf{Course Website:} \vspace{6pt}

All of the materials and assignments for the course will be posted
on the class website:
\begin{quote}
\url{https://statsmaths.github.io/stat389-s19}
\end{quote}
The website contains notes, assignment details, and supplemental materials.
At the end of the semester, this version of the course will be archived and
available for your reference.

\vspace{0.4cm}

\textbf{Labs:} \vspace{6pt}

During most class meetings, you will work on a series of assignments I refer
to as `Labs'. These may be a paper handout with questions or a code file that
requires that you fill in answers digitally. In order to succeed in the course
you should complete these prior to the next class meeting. Rather than
formally handing them in, you must instead fill out an online questionnaire
at some point prior to the next class meeting. I will not accept late
submissions. The questionnaire can be found through a link on the course
website. You are excused for forgetting to hand in \textbf{two}
questionnaires. Beyond these you will lose one point on your final grade for
every missing questionnaire.

\vspace{0.4cm}

\textbf{Exams:} \vspace{6pt}

We will have four exams during the semester. Exams will focus on
the material in each section of the course, but due to the cumulative nature
of the material each requires understanding previous sections. There may be
in-class and take-home components of each exam. The in-class portion will take
place on the following days:

\begin{itemize}\setlength\itemsep{0em}
\item 2019-02-05 (Tue)
\item 2019-02-26 (Tue)
\item 2019-03-26 (Tue)
\item 2019-04-16 (Tue)
\end{itemize}

Take-home components will typically be due in class the day on which the
in-class exam is assigned.

\vspace{0.4cm}

\textbf{Final Project:} \vspace{6pt}

In lieu of a final exam, the course concludes with a final project.
The project is due on the second to last class of the semester so that we can
accommodate in-class presentations during the final week. More details on the
project will be given prior to Spring Break.

\vspace{0.4cm}

\textbf{Final Grades:} \vspace{6pt}

You will receive a numeric score from 0-100 for each of the exams and the
final project. Your final numeric grade is determined by taking the average
score of your four best grades, rounded to the nearest integer. Finally,
subtract one point for every missing lab for which you failed to turn in a
form (beyond the grace window of two missing forms).

\medskip

The mapping from numeric grades to letter grades is given as follows:
\begin{itemize}\setlength\itemsep{0em}
\item[] \textbf{A} $\Rightarrow$ 90 to 100
\item[] \textbf{B} $\Rightarrow$ 80 to 89
\item[] \textbf{C} $\Rightarrow$ 70 to 79
\end{itemize}
I may assign pluses and minuses as needed. When appropriate, I may also modify
these cut-off scores to make them more generous (but will not make them more
strict).

\vspace{0.4cm}

\textbf{Attendance:} \vspace{6pt}

There is no formal attendance policy for this course. However, if you miss a
class it is your responsibility to catch up with the material. E-mail and
office hours are not a replacement for attendance. When you are present,
I expect you to arrive on time, engage with the material, and give us your
full attention.

\newpage

\textbf{Office Hours and Email:} \vspace{6pt}

There are three easy ways to get help with the course:

\begin{itemize}
\item \textbf{lab form}: You may ask questions as part of the lab form. This
is particularly useful if you have a general misunderstanding about a concept
or a very specific question about one of the lab questions.
\item \textbf{piazza}: I have created a Piazza site for our course; there is
a link available on the main course website. This free, private site allows
you to ask (optionally anonymous) questions visible to the entire class. I
will provide answers to any posted questions as soon as possible. Students may
also add their own responses.
\item \textbf{office hours}: You may also make an appointment to come to my
office hours. Please do this using the \texttt{calend.ly} link on the course
website. Appointments can be made up to a week in advance, but must be booked
within 24-hours of the time slot.
\end{itemize}

If you have personal issues of circumstance, feel free to email me directly.
For questions about the course material, however, please instead make a Piazza
post so that everyone can benefit from the answer.

\vspace{0.5cm}

\textbf{Class Policies:} \vspace{6pt}

The following class policies address some of the most common questions and
concerns that students have. If anything is unclear or not covered below,
please feel free to ask for clarification at any point in the semester.

\begin{itemize}\setlength\itemsep{0em}
\item \textbf{Academic honesty:} Cheating and plagiarism are grave scholarly
offenses and potential grounds
for expulsion; they are also a major barrier to your intellectual development.
You are expected to familiarize yourself with the entirety of the
University of Richmond’s Honor Code. If you are confused or unsure about
appropriate citation protocol or any other aspect of the Honor code,
please consult me before turning in an assignment.
\item \textbf{Special approval:} If you have special approval forms for extra
time on exams or any other circumstances I should know about, please speak
with me as early as possible so that we can best accommodate your needs.
\item \textbf{Late work:} You are expected to submit all work on time. The
final project will be accepted after the due date with a 10-point deduction
for each 24 hour period (rounded up) that it is late.
\item \textbf{Pass/Fail/Withdraw/Incomplete:} If you choose to take this
class pass/fail, a passing mark requires you to receive a grade of C- or
higher. I do not normally give a final grade of D. I am generally happy to
allow students to withdraw after the deadline, with approval of the Dean,
without penalty. However, I typically do not give a grade of incomplete (I or
Y) to students who have not finished at least three of the exams.
\item \textbf{Make-up exams:} There are no make-up exams. If you fail to
attend an exam without a valid excuse (given to me by email within 24-hours
of the exam) you will receive a score of zero. In the event of a valid reason
for missing the exam, the missing score will be filled in with the median
grade from the three remaining exams.
\item \textbf{Class conduct:} During class I expect you to refrain from checking
email, being on phones, or working on assignments for other classes.
\end{itemize}

\bigskip

\textbf{Notice:} \vspace{6pt}

I reserve the right to modify this syllabus, with advanced warning, throughout
the semester. If necessary, I will email the class list and post an updated
version of the document on the course website.


\end{document}





