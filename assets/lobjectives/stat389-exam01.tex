\documentclass[11pt]{article}
\usepackage[top=1.5cm,bottom=2cm,left=2cm,right= 2cm]{geometry}
\geometry{letterpaper}                   % ... or a4paper or a5paper or ...
%\geometry{landscape}                % Activate for for rotated page geometry
\usepackage[parfill]{parskip}    % Activate to begin paragraphs with an empty line rather than an indent
\usepackage{graphicx}
\usepackage{amssymb}
\usepackage{epstopdf}
\usepackage{amsmath}
\usepackage{multirow}
\usepackage{multicol}
\usepackage{changepage}
\usepackage{lscape}
\usepackage{enumitem}
\usepackage{ulem}
\DeclareGraphicsRule{.tif}{png}{.png}{`convert #1 `dirname #1`/`basename #1 .tif`.png}

\usepackage{xcolor}

\definecolor{oiB}{rgb}{.337,.608,.741}
\definecolor{oiR}{rgb}{.941,.318,.200}
\definecolor{oiG}{rgb}{.298,.447,.114}
\definecolor{oiY}{rgb}{.957,.863,0}

\definecolor{light}{rgb}{.337,.608,.741}
\definecolor{dark}{rgb}{.337,.608,.741}

\usepackage[colorlinks=false,pdfborder={0 0 0},urlcolor= dark,colorlinks=true,linkcolor=black]{hyperref}

\newcommand{\light}[1]{\textcolor{light}{\textbf{#1}}}
\newcommand{\dark}[1]{\textcolor{dark}{#1}}
\newcommand{\gray}[1]{\textcolor{gray}{#1}}

%\date{}                                           % Activate to display a given date or no date

%

\begin{document}

{\LARGE \textcolor{oiB}{Learning Objectives \hfill Exam 01}} \\

The first exam will mostly be comprised of lab questions, primarily with only
minor modifications or extensions. For your help preparing for the exam, here
are the concrete learning objects for the first unit of the course:

\begin{enumerate}
\renewcommand\labelenumi{\textcolor{light}{\textbf{LO \theenumi.}}}

\item Define and understand the role of \textbf{training data} and
\textbf{testing data} in statistical learning.

\item Utilize common \textbf{loss functions}, including mean squared error,
absolute error, and misclassification error. You should be able to evaluate
these by hand on small datasets and in R code from larger ones.

\item Apply the one-dimensional \textbf{best split} estimator by hand on small
datasets.

\item Understand the R code in the function \verb|casl_utils_best_split|.

\item Visualize simple linear regression from a scatter plot and understand
the interpretation of the coefficients.

\item Derive the simple linear regression ordinary least squares (OLS)
coefficients using calculus.

\item Apply the matrix format of the least squares estimator and understand
the notation for $y$, $X$, and $\beta$.

\item View a matrix as a linear transformation between $\mathbb{R}^n$ and
$\mathbb{R}^m$ and matrix multiplication as function composition.

\item Understand the matrix transpose and inverse, its notation, and rules
for applying these to matrix equations.

\item Apply the inner product and Euclidean-norm as matrix products between
column and row vectors.

\item Derive the equations for the gradient of an inner product:
\begin{align*}
\nabla_\beta \left( a^t \beta \right) &= a
\end{align*}
And the gradient of a quadratic form:
\begin{align*}
\nabla_\beta \left( \beta^t A \beta \right) &= A^t \beta + A \beta.
\end{align*}

\item Understand the concept of an orthogonal matrix and how it represents
rotations in $n$-dimensional space.

\item Apply the rules for gradient functions of vector and matrix forms to
derive the normal equations.

\item Derive properties of the ordinary least squares estimator such as
showing the residuals are perpendicular to the fitted values and that the
``hat'' matrix is a projection operator.

\item Understand the geometric interpretation of the singular value
decomposition (SVD) and role of the singular values.

\item Apply the SVD to the normal equations to find a computationally stable
solution to the ordinary least squares problem for linear regression.

\item Construct model matrices in R from a data frame object. Fit and evaluate
the ordinary least squares estimator using R's functions for matrix manipulation.

\item Compute the SVD in R and use matrix methods to directly compute the
ordinary least squares estimator.

\end{enumerate}

\end{document}

