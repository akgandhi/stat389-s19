%%%%%%%%%%%%%%%%%%%%%%%%%%%%%%%%%%%%%%%%%%%%%%%%%%%%%%%%%%%%%%%%%%%%%%%%%%%%%%
\documentclass[12pt,hidelinks]{article}

% 1. Load LaTeX packages
\usepackage{fontspec}
\usepackage{geometry}
\usepackage{lastpage}
\usepackage{fancyhdr}
\usepackage{hyperref}
\usepackage{amsmath}
\usepackage{amsthm}
\usepackage{xunicode}
\usepackage{listings}
\usepackage{color}
\usepackage{amssymb}

% 2. Define page dimensions and spacing
\geometry{top=1in, bottom=1in, left=1in, right=2in, marginparsep=4pt,
          marginparwidth=1in}
\setlength{\parindent}{0pt}
\setlength{\parskip}{12pt}

% 3. Set header, footer, and bibliography
\renewcommand{\headrulewidth}{0pt}
\pagestyle{fancyplain}
\fancyhf{}
\lfoot{}
\rfoot{page \thepage\ of \pageref{LastPage}}
\bibliographystyle{acm}

% 4. Set fonts for the document
\defaultfontfeatures{Mapping=tex-text}
\setromanfont{YaleNew}

% 5. Define custom code for book environments and commands
\DeclareMathOperator*{\argmin}{arg\,min}
\DeclareMathOperator*{\argmax}{arg\,max}
\newcommand{\code}[1]{\texttt{#1}}
\newcommand{\pkg}[1]{\textbf{#1}}

% 6. Define custom code for book environments and commands
\definecolor{verbgray}{gray}{0.9}
\definecolor{verbgray2}{gray}{0.975}

\lstnewenvironment{rcode}{%
  \lstset{backgroundcolor=\color{verbgray},
  frame=single,
  framerule=0pt,
  basicstyle=\ttfamily,
  keepspaces=true,
  columns=fullflexible}}{}

\lstnewenvironment{rres}{%
  \lstset{backgroundcolor=\color{verbgray2},
  frame=single,
  framerule=0pt,
  basicstyle=\ttfamily,
  keepspaces=true,
  columns=fullflexible}}{}

% 7. Define numbering scheme for equations (only needed for handout).
\numberwithin{equation}{section}
\setcounter{section}{8}

%%%%%%%%%%%%%%%%%%%%%%%%%%%%%%%%%%%%%%%%%%%%%%%%%%%%%%%%%%%%%%%%%%%%%%%%%%%%%%
\begin{document}

{\LARGE Lab Solutions 08}

\vspace*{12pt}

\textbf{1. Start by doing the text analysis questions in the Rmarkdown file.}

\vspace*{12pt}

See the solutions in the HTML file for Lab08.

\vspace*{12pt}

\textbf{2. Assume that a magical creature has told you all but the first
coefficient of $\beta^{LASSO}$ for some value of $\lambda$. How would you
go about solving for the first coefficient if you know all of the others?
Start by defining the partial residual:}
\begin{align}
r &= y - \sum_{j = 2}^p \beta_j \cdot X_j
\end{align}
\textbf{This is what is ``left over'' in y that we need to predict using the first
variable. Finding the best value for $\beta_1$ is surprisingly similar to the
uncorrelated columns case you derived last class. We want to minimize the
following quantity over $b_1$:}
\begin{align}
f(b_1) &= || y - X b ||_2^2 + 2 \lambda | b |_1 \\
&= || r - X_1 b_1 ||_2^2 + 2 \lambda \sum_j | b_j | \\
&= r^t r + X_1^t X_1 b_1^2 - 2 r^t X_1 b_1 + 2 \lambda \sum_j | b_j | \\
&\propto X_1^t X_1 b_1^2 - 2 r^t X_1 b_1 + 2 \lambda \cdot | b_1 |
\end{align}
\textbf{Note that $b_1$ is a scalar value, so that's why I have $b_1^2$ in the equation
and not $b_1^t b_1$. I've already done some of the work; now for your part:
derive a formula for $\beta_1$ by minimizing the function $f(b_1)$. You should
be able to carry through the method from last time almost exactly.}

\vspace*{12pt}

If $b_1 > 0$ then:

\begin{align}
f(b_1) &= X_1^t X_1 b_1^2 - 2 r^t X_1 b_1 + 2 \lambda \cdot b_1 \\
f'(b_1) &= 2 X_1^t X_1 b_1 - 2 r^t X_1 + 2 \lambda
\end{align}

Setting equal to zero:

\begin{align}
X_1^t X_1 b_1 &= r^t X_1 - \lambda \\
b_1 &= \frac{r^t X_1 - \lambda}{ X_1^t X_1}
\end{align}

Which gives a consistent value of $b_1 > 0$ if $r^t X_1 > \lambda$. If
$b_1 < 0$ the function becomes:

\begin{align}
f(b_1) &= X_1^t X_1 b_1^2 - 2 r^t X_1 b_1 - 2 \lambda \cdot b_1 \\
f'(b_1) &= 2 X_1^t X_1 b_1 - 2 r^t X_1 - 2 \lambda
\end{align}

Setting equal to zero:

\begin{align}
X_1^t X_1 b_1 &= r^t X_1 + \lambda \\
b_1 &= \frac{r^t X_1 + \lambda}{ X_1^t X_1}
\end{align}

Which gives a consistent value of $b_1 < 0$ if $r^t X_1 < -1 \lambda$. Putting
this together gives:

\begin{align}
\beta^{LASSO}_1 &= \begin{cases} \frac{r^t X_1 + \lambda}{ X_1^t X_1} & r^t X_1 \leq -\lambda \\
                               0 & |r^t X_1| \leq \lambda \\
                               \frac{r^t X_1 - \lambda}{ X_1^t X_1} & r^t X_1 \geq \lambda \end{cases}
\end{align}

Which looks a lot like the correlated case, with the partial residuals replacing the
value of $y$ from before.

\vspace*{12pt}

\textbf{3. There's nothing special above about the first coefficient. If we define
the partial residual as:}
\begin{align}
r &= y - \sum_{j \neq k} \beta_j \cdot X_j
\end{align}
\textbf{Your question also gives (replacing $X_1$ with $X_k$ and $b_1$ with $b_k$) a
formula for $\beta_k$ in terms of the other coefficients. Coordinate descent
is an optimization technique that minimizes a function $f$ over several variables
by optimizing the first variable assuming the others are fixed, then optimizing
the second variable assuming the others are fixed, and so on through convergence.
There is nothing specific that you need to do here, but understand how you now
have a formula for running coordinate decent on the $\ell_1$-penalized regression
problem.}

\vspace*{12pt}

As said, there is really nothing you need to do here other than understand the
technique.

\vspace*{12pt}

\textbf{4. Finally, input these formulas into the final section of the Rmarkdown
file.}

\vspace*{12pt}

See the solutions in the HTML file for Lab08.

\end{document}

