%%%%%%%%%%%%%%%%%%%%%%%%%%%%%%%%%%%%%%%%%%%%%%%%%%%%%%%%%%%%%%%%%%%%%%%%%%%%%%
\documentclass[12pt,hidelinks]{article}

% 1. Load LaTeX packages
\usepackage{fontspec}
\usepackage{geometry}
\usepackage{lastpage}
\usepackage{fancyhdr}
\usepackage{hyperref}
\usepackage{amsmath}
\usepackage{amsthm}
\usepackage{xunicode}
\usepackage{listings}
\usepackage{color}
\usepackage{amssymb}

% 2. Define page dimensions and spacing
\geometry{top=1in, bottom=1in, left=1in, right=2in, marginparsep=4pt,
          marginparwidth=1in}
\setlength{\parindent}{0pt}
\setlength{\parskip}{12pt}

% 3. Set header, footer, and bibliography
\renewcommand{\headrulewidth}{0pt}
\pagestyle{fancyplain}
\fancyhf{}
\lfoot{}
\rfoot{page \thepage\ of \pageref{LastPage}}
\bibliographystyle{acm}

% 4. Set fonts for the document
\defaultfontfeatures{Mapping=tex-text}
\setromanfont{YaleNew}

% 5. Define custom code for book environments and commands
\DeclareMathOperator*{\argmin}{arg\,min}
\DeclareMathOperator*{\argmax}{arg\,max}
\newcommand{\code}[1]{\texttt{#1}}
\newcommand{\pkg}[1]{\textbf{#1}}

% 6. Define custom code for book environments and commands
\definecolor{verbgray}{gray}{0.9}
\definecolor{verbgray2}{gray}{0.975}

\lstnewenvironment{rcode}{%
  \lstset{backgroundcolor=\color{verbgray},
  frame=single,
  framerule=0pt,
  basicstyle=\ttfamily,
  keepspaces=true,
  columns=fullflexible}}{}

\lstnewenvironment{rres}{%
  \lstset{backgroundcolor=\color{verbgray2},
  frame=single,
  framerule=0pt,
  basicstyle=\ttfamily,
  keepspaces=true,
  columns=fullflexible}}{}

% 7. Define numbering scheme for equations (only needed for handout).
\numberwithin{equation}{section}
\setcounter{section}{7}

%%%%%%%%%%%%%%%%%%%%%%%%%%%%%%%%%%%%%%%%%%%%%%%%%%%%%%%%%%%%%%%%%%%%%%%%%%%%%%
\begin{document}

{\LARGE Lab Solutions 07}

\vspace*{12pt}

\textbf{1. This was a question from last lab, but in case you did not get there,
make sure that you can derive a formula for ridge regression under the uncorrelated
assumption.}

\vspace*{12pt}

In the previous lab, I assumed that $X^t X$ was equal to $n$ times the identity.
Removing the $n$ term gives:

\begin{align*}
\beta^{RIDGE} &= \frac{1}{1 + \lambda} \cdot X^t y = \frac{1}{1 + \lambda} \cdot \beta^{OLS}.
\end{align*}

See the previous lab solutions for the details.

\vspace*{12pt}

\textbf{2. Write the best subset selection loss function as a sum over $j$ from $j=1$
to $j=p$. That is, you can write the loss function as a sum of independent terms,
where each elements depends only on the $j$'th column of $X$ and the $j$'th component
of $\beta$.}

\vspace*{12pt}

Writing $\chi\{ b_j \neq 0 \}$ as the indicator function that is one when the
statement is true and zero otherwise, we have:

\begin{align*}
|| y - X b ||_2 + \lambda || b ||_0 &=
y^t y + b^t X^t X b - 2 y^t X b + \lambda \cdot \sum_j \chi\{ b_j \neq 0 \} \\
&= y^t y + b^t b - 2 y^t X b + \lambda \cdot \sum_j \chi\{ b_j \neq 0 \} \\
&= y^t y + \sum_j b_j^2 - 2 \sum_j y^t X_j b_j + \lambda \cdot \sum_j \chi\{ b_j \neq 0 \} \\
&= y^t y + \sum_j \left[ b_j^2 - 2  y^t X_j b_j + \lambda \cdot \chi\{ b_j \neq 0 \} \right]
\end{align*}

Which is now a decoupled sum over the components of $\beta_j$, plus a leading
constant term that doesn't effect the loss.

\vspace*{12pt}

\textbf{3. Repeat the previous question for lasso regression, showing that it also
decouples over the individual variables.}

\vspace*{12pt}

The lasso regression works similarly:

\begin{align*}
|| y - X b ||_2 + 2 \lambda || b ||_1 &=
y^t y + b^t X^t X b - 2 y^t X b + 2 \lambda \cdot \sum_j | b_j | \\
&= y^t y + b^t b - 2 y^t X b + 2 \lambda \cdot \sum_j | b_j | \\
&= y^t y + \sum_j b_j^2 - 2 \sum_j y^t X_j b_j + 2 \lambda \cdot  \sum_j | b_j | \\
&= y^t y + \sum_j \left[ b_j^2 - 2 y^t X_j b_j + 2 \lambda \cdot | b_j |\right]
\end{align*}

The only difference being the last term in the equation.

\vspace*{12pt}

\textbf{4. Understand why your answers to the last two questions allow us to minimize
the loss function individually for each component $\beta_j$.}

\vspace*{12pt}

In general if we want to minimize a function $f(x, y) = f_1(x) + f_2(y)$ over
$x$ and $y$, it should be clear that all we need to do is minimize $f_1$ over x
and $f_2$ over y. The same logic applies to any number of variables.

\vspace*{12pt}

\textbf{5. Assume that $\beta_j \neq 0$. What would be its optimal value under best
subset regression? When is the loss at this point better than setting $\beta_j = 0$?
Put these conditions together to get a general formula for $\beta_j$ under best
subset selection.}

\vspace*{12pt}

We want to minimize the quantity:

\begin{align*}
f_j(b_j) &= b_j^2 - 2  y^t X_j b_j + \lambda \cdot \chi\{ b_j \neq 0 \}
\end{align*}

If $\beta_j \neq 0$ then, this is equal to a simple quadratic function in $b_j$:

\begin{align*}
b_j^2 - 2  y^t X_j b_j + \lambda
\end{align*}

We minimize this just like any other quadratic, by taking the derivative with
respect to $b_j$ and setting it equal to zero:

\begin{align*}
2 b_j - 2  y^t X_j &= 0 \\
b_j &= y^t X_j
\end{align*}

The value of $f_j$ at this point is:

\begin{align*}
f_j(y^t X_j) &= (y^t X_j)^2 - 2  (y^t X_j) \cdot (y^t X_j) + \lambda \\
&= \lambda - (y^t X_j)^2
\end{align*}

The question is, when will this be better than the alternative of setting $b_j = 0$?
There we have $f_j(0) = 0$, which will be better---more negative---whenever
$\lambda \geq (y^t X_j)^2$. This should seem reasonably because we are more likely to
set a coefficient equal to zero if (i) $\lambda$ is large, or (ii) the correlation
$y^t X_j$ is small.

Putting this together, we can compactly write:

\begin{align*}
\beta^{BSR} &= \begin{cases} 0 & \lambda > (y^t X_j)^2 \\
                             y^t X_j & \text{else} \\ \end{cases}
\end{align*}

Or, even better, as:

\begin{align*}
\beta^{BSR} &= \begin{cases} 0 & \lambda > (\beta^{OLS}_j)^2 \\
                             \beta^{OLS}_j & \text{else} \\ \end{cases}
\end{align*}

Which shows the direct link with the unpenalized solution.

\vspace*{12pt}

\textbf{6. Use a similar approach to find a solution for $\beta_j$ for lasso regression.
This requires a little bit more work but is very doable with simple one-dimensional
calculus!}

\vspace*{12pt}

We want to minimize the quantity:

\begin{align*}
f_j(b_j) &= b_j^2 - 2  y^t X_j b_j + 2 \lambda \cdot | b_j |
\end{align*}

Assume to start that the optimal value of $b_j$ is greater than zero. Then,
$| b_j | = b_j$ and we only have to consider points of $f_j$ that can be written
as a simply quadratic function:

\begin{align*}
f_j(b_j) &= b_j^2 - 2  y^t X_j b_j + 2 \lambda  b_j
\end{align*}

The optimal value occurs when the derivative is equal to zero, which happens at:

\begin{align*}
f_j'(b_j) &= 2 b_j - 2  y^t X_j + 2 \lambda \\
0 &= b_j - y^t X_j + \lambda \\
b_j &= y^t X_j - \lambda.
\end{align*}

Great! Except, this is only valid when $b_j > 0$. Therefore, this solution only
works when $y^t X_j > \lambda$. Now, assume instead that $f_j$ is optimized when
$b_j < 0$. Now it reduces to:

\begin{align*}
f_j(b_j) &= b_j^2 - 2  y^t X_j b_j - 2 \lambda  b_j
\end{align*}

And setting the derivative equal to zero yields:

\begin{align*}
f_j'(b_j) &= 2 b_j - 2  y^t X_j - 2 \lambda \\
0 &= b_j - y^t X_j - \lambda \\
b_j &= y^t X_j + \lambda.
\end{align*}

This, in turn, only holds if $y^t X_j \leq -\lambda$. So, what happens if
$|y^t X_j| \leq \lambda$? Neither of these conditions hold, and $f_j$ is optimized
when $b_j = 0$. Putting this all together yields:

\begin{align*}
\beta^{LASSO} &= \begin{cases} y^t X_j + \lambda & y^t X_j \leq -\lambda \\
                               0 & |y^t X_j| \leq \lambda \\
                               y^t X_j - \lambda & y^t X_j \geq \lambda \end{cases}
\end{align*}

Or, very compactly, as:

\begin{align*}
\beta^{LASSO} &= \begin{cases} \beta^{OLS}_j - \lambda \cdot sign(\beta^{OLS}_j) & |y^t X_j| \geq \lambda \\
                                0 & \text{else} \end{cases}
\end{align*}

In words, the ordinary least squares coefficients are shrunk towards zero by a
linear factor of $\lambda$. Coefficients with absolute size less than $\lambda$
are set to zero.

\vspace*{12pt}

\textbf{7-9. Draw a sketches for the three estimators}

\vspace*{12pt}

Look at the R output for an example of the ridge and lasso estimators. The plot of
the best-subset regression is straightforward if you have the correct equation; it
is a piecewise constant function.

\end{document}

